%-----------------------------------------------------------------------------%
\chapter{\babTiga}
Pada bab ini berisi diagram alir penelitian yang dilakukan dan perancangan simulasi....
%\begin{center}
%	\textbf{BAB III}
%	\par \textbf{PERENCANAAN SISTEM}
%\end{center}
%-----------------------------------------------------------------------------%
% \section{Diagram Alir Penelitian}

% %\begin{flushleft}
% %	\textbf{3.1 Diagram alir penelitian}
% %\end{flushleft}
% %\vspace{-0.5 cm}
% \begin{figure}
% 	\centering
% 	\includegraphics[width=0.9\linewidth]{"pics/TANat/diagram alir baru2"}
% 	\caption{Diagram alir simulasi.}
% 	\label{fig:diagram-alir-baru}
% \end{figure}


% \vspace{-0.5 cm}
% \section{Parameter Input}
% Perancangan simulasi sistem VLC menggunakan beberapa parameter input yang akan digunakan dalam perhitungan. Berikut parameter input pada perhitungan seperti Tabel 3.1.
% \par
% \begin{table}[h]
% 	\centering
% 	\caption{Parameter input sistem.}
% 	\begin{tabular}{|l|l|l|l|}
% 		\hline
% 		\rowcolor[HTML]{E7E6E6} 
% 		No.                  & \multicolumn{2}{l|}{\cellcolor[HTML]{E7E6E6}Parameter}       & Spesifikasi                                                                     \\ \hline
% 		1.                   & \multicolumn{2}{l|}{Ukuran Ruangan}                          & 5x5x3 $m^2$                                                      \\ \hline
% 		&                                        & Jenis               & LED                                                                             \\ \cline{3-4} 
% 		&                                        & Jumlah              & 3                                                                               \\ \cline{3-4} 
% 		&                                        & Daya                & 1 Watt                                                                          \\ \cline{3-4} 
% 		&                                        & Variasi Koordinat 1 & \begin{tabular}[c]{@{}l@{}}(0,5, 4,5, 3)\\ (2,5, 0,5, 3)\\ (4,5, 4,5, 3)\end{tabular} \\ \cline{3-4} 
% 		&                                        & Variasi Koordinat 2 & \begin{tabular}[c]{@{}l@{}}(1,5 , 3,5, 3)\\ (2,5, 1,5, 3)\\ (3,5, 3,5, 3)\end{tabular} \\ \cline{3-4} 
% 		&                                        & Variasi Koordinat 3 & \begin{tabular}[c]{@{}l@{}}(2, 2,5, 3)\\ (2,5, 2, 3)\\ (3, 2,5, 3)\end{tabular}       \\ \cline{3-4} 
% 		\multirow{-7}{*}{2.} & \multirow{-7}{*}{\textit{Transmitter}} & FWHM                & $70^\circ$                                                                      \\ \hline
% 		&                                        & Jenis               & \textit{PIN Photodiode}                                                         \\ \cline{3-4} 
% 		&                                        & Jumlah              & 3                                                                               \\ \cline{3-4} 
% 		&                                        & Variasi Koordinat 1 & \begin{tabular}[c]{@{}l@{}}(2, 3, 0)\\ (2,5, 2, 0)\\ (3, 3, 0)\end{tabular}         \\ \cline{3-4} 
% 		&                                        & Variasi Koordinat 2 & \begin{tabular}[c]{@{}l@{}}(1,5, 3,5, 0)\\ (2,5, 1,5, 0)\\ (3,5, 3,5, 0)\end{tabular} \\ \cline{3-4} 
% 		&                                        & Area Detektor       & $10^{-4} \text{m}^2$                                                \\ \cline{3-4} 
% 		\multirow{-6}{*}{3.} & \multirow{-6}{*}{\textit{Receiver}}    & FOV                 & $60^\circ$                                                                      \\ \hline
% 	\end{tabular}
% \end{table}
% \par
% \vspace{15cm}
% \section{Perhitungan}
% Pada subbab ini memaparkan hasil perhitungan untuk mendapatkan hasil dari parameter performansi sistem yang dihasilkan pada proses simulasi...