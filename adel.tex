\documentclass{article}
\usepackage{graphicx}
\begin{document}
	\begin{table}
	\centering
	\begin{tabular}{|c|p{4cm}|} \hline
		 \includegraphics[width=0.7\linewidth]{pics/01.pdf} & buka file dengan template latex yang sudah di download.  pastikan latex sudah di install dengan baik dan dapat digunakan.\\ \hline
		 
		  \includegraphics[width=0.7\linewidth]{pics/02.pdf}& buka file latex yang terdapat pada template yang sudah disediakan dengan nama filenya adalah skripsi seperti yang terlihat pada gambar, terdapat folder yang telah dilingkari\\ \hline
		
		\includegraphics[width=0.7\linewidth]{pics/03.pdf} & setelah file tersebut dibuka  akan terlihat seperti pada  gambar disamping \\ \hline
		
	\end{tabular}
\end{table}

\begin{table}
\centering
\begin{tabular}{|c|p{4cm}|} \hline
	
		\includegraphics[width=0.7\linewidth]{pics/05.pdf} & pada bagian ini dimulai  dari menuliskan judul yang sudah  di siapkan seperti contoh  yang ada pada gambar. setelah itu yang terdapat  pada "awal konfigurasi" ini juga  bukan hanya judul, pada bagian "awal konfigurasi"  terdapat biodata yang harus diisi seperti nama dosen  pembimbing yang akan  ada pada lembar pengesahan. \\ \hline
		
		\includegraphics[width=0.7\linewidth]{pics/04.pdf} & klik pada bagian yang  bernama "sampul" untuk mengatur  hal-hal yang terdapat  pada cover. seperti mengatur letak gambar  atau jarak antar judul dengan tulisan lainnya karena jika tidak diatur tulisan yang terdapat  pada cover akan kelembar  berikutnya karena setiap orang akan memiliki judul yang berbeda dan pengaturan pada cover yang berbeda pula. \\ \hline
		
			\end{tabular}
		\end{table}
	\begin{table}
		\centering
		\begin{tabular}{|c|p{4cm}|} \hline
		
		\includegraphics[width=0.7\linewidth]{pics/06.pdf} & pada bagian ini akan dijelaskan bagaimana cara untuk menampilkan bab dan menutup bab apabila belum digunakan pasa saat itu. seperti contoh pada gambar disamping disini semua bab masih ditampilkan. bab yang akan dipakai hanya bab 1, 2, dan 3. bab yang akan dihilangkan adalah bab 4 dan bab 5. \\ \hline
		
		\includegraphics[width=0.7\linewidth]{pics/07.pdf} & untuk menghilangkan bab 4 dan 5 cara pertama yang harus dilakukan adalah klik bagian yang bernama skripsi seperti yang sudah dilingkari pada gambar lalu akan muncul seperti yang terdapat digambar sampingnya. lihat yang sudah diberi tanda persegi, dalam persegi tersebut terdapat konfigurasi pada setiap bab. agar bab 4 dan bab 5 hilang maka gunakan "CTRL+T" di setiap bab yang akan dihilangkan seperti gambar dibawah ini. \\ \hline
	\end{tabular}
\end{table}
\begin{table}
		\begin{tabular}{|c|p{4cm}|} \hline


		\includegraphics[width=0.7\linewidth]{pics/08.pdf} & telihat pada gambar yang diberi tanda persegi pada bab 4 dan bab 5 terdapat tanda persen yang berguna untuk menutup bab tersebut yang terletak sebelum tanda slice. jika ingin menampilkan kembali bab tersebut cukup dengan menghapus tanda persen tersebut atau dengan menggunakan "CTRL+U" pada bab yang akan di tampilkan. \\ \hline
		
	\end{tabular}
\end{table}
\end{document}