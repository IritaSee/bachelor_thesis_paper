
\chapter{\babSatu}
%\begin{center}
%	\textbf{BAB I}
%\par \textbf{PENDAHULUAN}
%\end{center}
%-----------------------------------------------------------------------------%
\section{Latar Belakang Masalah}
%\textbf{1.1 Latar Belakang Masalah}
%-----------------------------------------------------------------------------%
Kendaraan udara tanpa awak, atau \textit{Unmanned Aerial Vehicles} (UAV) telah dilengkapi dengan berbagai instrumentasi. Berbagai instrumen tersebut digunakan untuk navigasi kendaraan itu sendiri, atau untuk pengambilan data dari jarak jauh. Pengambilan data menggunakan UAV sudah semakin sering dilakukan seiring perkembangan teknologi pada UAV dan meningkatnya kebutuhan pengambilan data dari tempat yang sulit dijangkau.

\par Salah satu instrumen yang populer digunakan pada UAV adalah kamera dengan resolusi tinggi. Kamera pada UAV ini dapat mengambil foto atau video yang disebut citra \textit{aerial}. Citra \textit{aerial} dapat diimplementasikan dalam berbagai hal. Pemantauan dan pengindraan jarak jauh \cite{ricecounting} \cite{rs4113390}, mitigasi, dan pemetaan pasca bencana merupakan macam implementasi yang sering melibatkan citra dari UAV \cite{uavdesigndeploy} \cite{6237316}.  Citra \textit{aerial} memiliki berbagai karakteristik diantaranya (i) \textit{ultra-high spatial resolution} \cite{app9040643}, (ii) dipengaruhi oleh kondisi cuaca, dan (iii) dipengaruhi oleh ketinggian. 

\par Model jaringan syaraf tiruan dapat membatu implementasi penggunaan citra UAV \cite{kyrkou2018dronet}. Model jaringan syaraf tiruan dapat dibentuk untuk mengenali obyek-obyek, dan \textit{EfficientDet} adalah salah satu dari model pengenalan obyek yang tersedia. \textit{EfficientDet} lebih ringan secara komputasi dibandingkan model pengenalan obyek yang lain \cite{tan2020efficientdet}. Model jaringan syaraf tiruan khususnya pengenalan obyek kendaraan dan manusia yang tersedia tidak dilatih menggunakan citra UAV, maka model yang tersedia tidak dihadapkan dengan berbagai kondisi unik dari karakteristik citra UAV, dan akan menghasilkan prediksi yang buruk. Bila model deteksi obyek yang sudah ada dilatih kembali menggunakan citra UAV, maka akan memakan beban komputasi yang tinggi karena tingginya resolusi dari citra UAV. Maka dari itu, penulis ingin meneliti hasil deteksi manusia dan kendaraan dari citra UAV menggunakan model \textit{EfficientDet} yang rendah beban komputasinya.

% \section{Rumusan Masalah}
% Adapun permasalahan yang terjadi yaitu....
% \section{Tujuan dan Manfaat}
% Tujuan dari Tugas Akhir ini adalah .... Adapun manfaat dalam Tugas Akhir ini adalah:
% \begin{enumerate}
% 	\item Mengetahui...
% 	\item Dapat...
% 	\item Mendapatkan...
% \end{enumerate}
% \section{Batasan Masalah}
% Batasan masalah untuk membatasi penelitian ini adalah :
% \begin{enumerate}
% 	\item Simulasi hanya menggunakan jenis kanal \textit{Line Of Sight} (LOS).
% 	\item Tanpa interferensi dari cahaya lain dan sumber cahaya hanya dari LED.
% 	\item .....
% \end{enumerate}

% \section{Metode Penelitian}
% Metode penelitian yang diterapkan dalam penyelesaian Tugas Akhir ini dengan melakukan simulasi dan perhitungan....

% %% buat PKIP pakenya tabel milestone
% \section{Jadwal Pelaksanaan}
% %\section{Sistematika Penulisan}
% Sistematika penulisan Tugas Akhir ini adalah sebagai berikut :
% \begin{itemize}
% 	\item \textbf{BAB II DASAR TEORI} \par
% 	Bab ini membahas landasan teori dan literatur yang digunakan dalam proses penelitian analisis akurasi sistem VLC pada penentuan posisi objek di dalam ruangan.
% 	\item \textbf{BAB III PERANCANGAN SISTEM} \par
% 	Bab ini berisi tahapan-tahapan yang dilakukan dalam proses penelitian berupa diagram alir penelitian, parameter yang menjadi referensi penetian, dan desain rancangan setiap skenario.
% 	\item \textbf{BAB IV ANALISIS SIMULASI SISTEM} \par
% 	Bab ini berisi  pembahasan hasil dari nilai \textit{positioning error} dan akurasi setiap variasi skenario. Pada bab ini juga disertakan tabel dan grafik untuk mempermudah proses analisis.
% 	\item \textbf{BAB V KESIMPULAN DAN SARAN} \par
% 	Bab ini berisi kesimpulan dan saran Tugas Akhir untuk pengembangan selanjutnya.
% \end{itemize}







